
% Mostrar hasta paragraph (cuarto nivel) en la tabla de contenido
\setcounter{tocdepth}{5}
\setcounter{secnumdepth}{5}

% Personalizar paragraph para que actúe como una mini-sección
\titleformat{\paragraph}
  {\normalfont\normalsize\bfseries} % estilo
  {\theparagraph}                   % numeración
  {1em}                             % espacio entre número y título
  {}                                % código antes del título

% Asegura que los paragraph tengan su propia línea
\titlespacing*{\paragraph}{0pt}{1.5ex plus 0.5ex minus .2ex}{1em}

%----------------------------------------------------------------------------
\begin{doublespace} 

    % Definición de colores personalizados
    \definecolor{lightgray}{rgb}{0.9, 0.9, 0.9}
    \definecolor{lightblue}{rgb}{0.8, 0.9, 1.0}
    \definecolor{aliceblue}{rgb}{0.94, 0.97, 1.0}
    \definecolor{bleudefrance}{rgb}{0.19, 0.55, 0.91}

    \renewcommand{\arraystretch}{1.6} % Espaciado entre filas


\section{SPRINT N° 1}
\subsection{TAREAS A REALIZAR DURANTE EL SPRINT PLANNING (R-2)}

% Usamos \addcontentsline para incluir el paragraph en el índice
\paragraph{\Large\textbf {Historia de Usuario (propone el Product Owner)}}

\subfile{1/R-2/a/Historia_de_usuario}

\clearpage  %nueva pagina
% Usamos \addcontentsline para incluir el paragraph en el índice
\paragraph{\Large\textbf {Objetivo del Sprint (definido por el Scrum Máster)}}
\addcontentsline{toc}{paragraph}{Objetivo del Sprint}
\subfile{1/R-2/b/Objetivo_sprint_1}

%nueva pagina
\paragraph{\Large\textbf {Planning Poker (estimación de Puntos)}}
\addcontentsline{toc}{paragraph}{Planning Poker}
\subfile{1/R-2/c/Planning_poker}

\clearpage  %nueva pagina
\paragraph{\large\textbf {SELECCIÓN DE HISTORIAS DE USUARIOS (desarrolladores)}}
\addcontentsline{toc}{paragraph}{selecciona de Historias de Usuario}
\subfile{1/R-2/d/HU_a_desarrollar}

\clearpage  %nueva pagina
\paragraph{\Large\textbf {DISEÑO DE LA SOLUCIÓN (desarrolladores)}}
\addcontentsline{toc}{paragraph}{Diseño de la Solución}
\subfile{1/R-2/e/Diseño_solucion}

\clearpage %nueva pagina
\paragraph{\Large\textbf {SPRINT BACKLOG}}
\addcontentsline{toc}{paragraph}{Sprint Backlog}
\subfile{1/R-2/f/Sprint_Backlog}


%---------------------------------------------------------------------------

\clearpage  %nueva pagina
%TAREAS A REALIZAR DURANTE EL SPRINT PLANNING (R-3)- NUESTRO SCRUM

\subsubsection{ACTIVIDADES A REALIZAR DURANTE LA EJECUCION DEL SPRINT (R-3)}

\paragraph{\Large\textbf {Diseño de la arquitectura de software}}
\addcontentsline{toc}{paragraph}{diseño de la arquitectura de software}
\subfile{1/R-3/D_H_U_seleccion/D_H_U_seleccionada}


%---------------------------------------------------------------------------
\clearpage  %nueva pagina
\begin{landscape}

    \subsubsection*{\Large\textbf{SCRUM DIARIO (DAILY SCRUM)}}
    \addcontentsline{toc}{paragraph}{SCRUM DIARIO (DAILY SCRUM)}
    \subfile{1/R-3/daily_scrum}

\end{landscape}



%---------------------------------------------------------------------------
\clearpage  %nueva pagina
\subsubsection{REVISION DE SPRINT (R-4)}

\subfile{1/R-4/revision_sprint}




%---------------------------------------------------------------------------
\subsubsection{RETROSPECTIVA DE SPRINT (R-5)}
\subfile{1/R-5/retrospectiva_sprint}

\end{doublespace}